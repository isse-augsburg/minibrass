\begin{frame}{Motivation}
We could keep adding more and more desirable constraints

\begin{itemize}
\item ``Not only would I like to stand next to a friend, but Mike makes me look better than Joe.''
\item ``This robot should not do the drilling task as its driller is almost broken.''
\item ``This module should be deployed to server $A$ as its required dependencies are already there.''
\end{itemize}

\vspace*{1ex} \pause 
but at some point, problems become \alert{unsatisfiable}.

\vspace*{2ex} \pause 

\begin{itemize}
\item[-] $\rightarrow$ returning \texttt{UNSAT} is not an option in autonomic computing settings! \pause 
\item[-] We want \emph{graceful} degradation: satisfy most 
constraints (maybe even distinguish them)
\end{itemize}

\end{frame}

\begin{frame}
\frametitle{Over-Constrained Problems}

\vspace*{1ex}

Consider: 

\vspace*{2ex}

$x, y, z \in \{1,2,3\}$ with 
\bgroup\abovedisplayskip4pt\belowdisplayskip4pt
\begin{align*}
  \mathrm{c}_1 &: x + 1 = y
\\[-.4ex]
  \mathrm{c}_2 &: z = y + 2
\\[-.4ex]
  \mathrm{c}_3 &: x + y \leq 3
\end{align*}
\egroup

\begin{itemize}
  \item Not all constraints can be satisfied simultaneously 
\begin{itemize} \pause
  \item e.\,g., $\mathrm{c}_2$ forces $z = 3$ and $y = 1$, making $\mathrm{c}_1$ impossible
\end{itemize}

  \item We \cemph{choose} between solutions that satisfy either $\{ \mathrm{c}_1, \mathrm{c}_3 \}$ or $\{ \mathrm{c}_2, \mathrm{c}_3 \}$.
\end{itemize}

\vspace*{2ex}

Which solutions $v \in [X \to D]$ are \alert{preferred}?

\end{frame}

\begin{frame}
\frametitle{Constraint Relationships}

Approach~\cite{Schiendorfer13}
\begin{itemize}
  \item Define preference relation over constraints $C$ to denote which constraints are \alert{more important}, e.\,g.
\begin{itemize}
  \item $c_1$ is more important than $c_2$

  \item $c_1$ is more important than $c_3$
\end{itemize}

\vspace*{2ex}
\item How \textbf{much} more important is $c_1$ ?
\begin{itemize}
\item More important than only either of $c_2$ or $c_3$?
\item More important than $c_2$ and $c_3$ \emph{combined}?
\end{itemize}
\end{itemize}


\begin{textblock*}{2.5cm}[1,1](\textwidth-0.5cm,\textheight-3.63cm)
\begin{tikzpicture}[auto,
                    ->,>=stealth',shorten >=1pt,thick,
                    node distance=.7cm,inner sep=2pt,
                    constraint/.style={circle,fill=black!15,draw,font=\sffamily\small}]
\node[constraint node] (1) at (0, 0)                   {$\mathrm{c}_1$};
\node[constraint node] (2) at ($ (1) + (-0.8, -0.8) $) {$\mathrm{c}_2$};  
\node[constraint node] (3) at ($ (1) + ( 0.8, -0.8) $) {$\mathrm{c}_3$};  
%  
\path[every node/.style={font=\sffamily\tiny}]
  (2) edge (1)
  (3) edge (1)
  ;
\end{tikzpicture}
\end{textblock*}

%\vspace*{2ex}
%\begin{small}
%A.~Schiendorfer, J.-Ph.~Steghöfer, A.~Knapp, F.~Nafz, W.~Reif (2013)
%\end{small}
\end{frame}

\begin{frame}{Example: Rostering}
Consider again our rostering example 
\begin{center}
\begin{tabular}{|c|c|c|c|c|c|}
\hline 
$\mathsf{worksShift}$ & Monday & Tuesday & Wednesday & Thursday & Friday \\ 
\hline 
Nurse 1 & off & 1 & 2 & 3 & 2 \\ 
Nurse 2 & 1 & off & 1 & 1 & 1 \\ 
Nurse 3 & 1 & 2 & off & 1 & 2 \\ 
%Nurse 4 & 2 & 2 & 2 & off & 3 \\ 
%Nurse 5 & 3 & 3 & 2 & 3 & off \\ 
\hline 
\end{tabular} 
\end{center}%

\vspace*{.8cm}

\begin{center}
\begin{tikzpicture}[auto,
                    ->,>=stealth',shorten >=1pt,thick,
                    node distance=.7cm,inner sep=4pt,
                    constraint/.style={rectangle,rounded corners, inner sep = 3pt, fill=black!15,draw,font=\sffamily\small}]
\node[constraint] (1) at (0, 0)                   {$\mathsf{fairNightShifts}$};
\node[constraint] (2) at ($ (1) + (-1.8, -1.2) $) {$\mathsf{nurseOneMonday}$};  
\node[constraint] (3) at ($ (1) + ( 1.8, -1.2) $) {$\mathsf{nurseOneTwoDays}$};  

\onslide<2->{  
\node[overlay,align=left,rectangle callout,%
      callout absolute pointer=(1.east),fill=isseorange!50] (bubble) at (2.2,0.8) {$\forall \texttt{n in NURSE}~\exists \texttt{d in DAYS} : \mathtt{worksShift}[n,d] = 3$}; }

\onslide<3->{  
\node[overlay,align=left,rectangle callout,%
      callout absolute pointer=(2.north),fill=isseorange!50] (bubble) at (-4.0,0.1) {\texttt{worksShift[n1,mo] = 0}}; 
}

\onslide<4->{  
\node[overlay,align=left,rectangle callout,%
      callout absolute pointer=(3.south),fill=isseorange!50] (bubble) at (1.6,-2.4) {$\exists \texttt{d in DAYS} : \texttt{worksShift[n1,d] = 0}~\wedge $\\ $\texttt{  worksShift[n1,d+1] = 0}$};
      }
%  
\path[every node/.style={font=\sffamily\tiny}]
  (2) edge (1)
  (3) edge (1)
  ;
\end{tikzpicture}
\end{center}

\begin{textblock*}{2.5cm}[1,1](\textwidth-2cm,\textheight-7.03cm)
\includegraphics[width=\textwidth]{img/nurse.jpg}
\end{textblock*}

\end{frame}

\begin{frame}{Combining Preferences}
\input{img/synth-graph-no-weights}

\vspace*{2ex}

\begin{itemize}
\item ``Organizational'' constraint should be considered most important
\item \cemph{No agent} (i.e., no agent's constraint) should be preferred (unbiased combination).
\end{itemize}
\end{frame}


\begin{frame} \small
\frametitle{Ordering Solutions}

Two rules for an \alert{isWorseThan} ordering (called \emph{single-predecessor-dominance}, SPD):
%
\bgroup\abovedisplayskip=4pt\belowdisplayskip=12pt
\begin{gather*}
  V \uplus \{ c \}  \supset_{\mathsf{SPD}}  V 
\\
  V \uplus \{ c_{\mathrm{gold}} \}  \supset_{\mathsf{SPD}}  V \uplus \{ c_{\mathrm{silver}} \}
\quad\text{if $c_{\mathrm{silver}}$ is less important than $c_{\mathrm{gold}}$}
\end{gather*}
\egroup

\begin{center}
\begin{tikzpicture}[auto,
                    ->,>=stealth',shorten >=1pt,thick,
                    node distance=1cm,inner sep=2pt,
                    constraint/.style={circle,fill=black!15,draw,font=\sffamily\small}]
\begin{scope}
\node[constraint,fill=green] (1) at (0, 0)                   {$\mathrm{c}_7$};
\node[constraint,fill=red] (2) at ($ (1) + (-1.2, -0.8) $) {$\mathrm{c}_4$};  
\node[constraint,fill=green]   (3) at ($ (2) + (0.0, -0.9) $) {$\mathrm{c}_5$};  
\node[constraint,fill=green] (4) at ($ (3) + ( 0.0, -1.0) $) {$\mathrm{c}_6$};
\node[constraint,fill=green] (7) at ($ (1) + ( 1.2, -0.8) $) {$\mathrm{c}_1$};  
\node[constraint,fill=red]   (8) at ($ (7) + (-0.6, -0.9) $) {$\mathrm{c}_2$};  
\node[constraint,fill=red]   (9) at ($ (7) + ( 0.6, -0.9) $) {$\mathrm{c}_3$};  
%  
\path[every node/.style={font=\sffamily\tiny}]
  (2) edge (1)
  (7) edge (1)
  (3) edge (2)
  (4) edge (3)
  (8) edge (7)
  (9) edge (7)
  ;
  
  
\node (solA) at (0,-3.5) {Solution $A$};
\end{scope}
%
\begin{scope}[xshift=6.3cm]
\node[constraint,fill=green] (1) at (0, 0)                   {$\mathrm{c}_7$};
\node[constraint,fill=green]   (2) at ($ (1) + (-1.2, -0.8) $) {$\mathrm{c}_4$};  
\node[constraint,fill=red] (3) at ($ (2) + (0.0, -0.9) $) {$\mathrm{c}_5$};  
\node[constraint,fill=green] (4) at ($ (3) + ( 0.0, -1.0) $) {$\mathrm{c}_6$};  
\node[constraint,fill=green] (7) at ($ (1) + ( 1.2, -0.8) $) {$\mathrm{c}_1$};  
\node[constraint,fill=red]   (8) at ($ (7) + (-0.6, -0.9) $) {$\mathrm{c}_2$};  
\node[constraint,fill=red]   (9) at ($ (7) + ( 0.6, -0.9) $) {$\mathrm{c}_3$};  
%  
\path[every node/.style={font=\sffamily\tiny}]
  (2) edge (1)
  (7) edge (1)
  (3) edge (2)
  (4) edge (3)
  (8) edge (7)
  (9) edge (7)
  ;
  
\node (solB) at (0,-3.5) {Solution $B$};
\end{scope}
%
\node [align=center] (R) at (3.45, -1.4) {
$ \supset_{\mathsf{SPD}}$ \\ ``is worse than''
};

\end{tikzpicture}
\end{center}

%\begin{itemize}
%\item Known as \emph{Smyth ordering} in the area of powerdomains \hfill \cite[Ch.~9]{amadio-curien:1998}
%\item Entsteht aus \emph{freier Konstruktion} über Constraint-Relationship.\hfill \cite{knapp-schiendorfer2014ictai}
%\end{itemize}

%
%\cemph{Ordnungs}relation über Zuweisungen 
%\bgroup\abovedisplayskip4pt
%\begin{equation*}
%  w \SPDord{R} v \iff \{ c \in C \mid v \not\models c \} \mathrel{({\SPDrel{R}})^{+}} \{ c \in C \mid w \not\models c \}
%\end{equation*}
%\egroup

\end{frame}