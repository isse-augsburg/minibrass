\begin{frame}{Motivation}
We could keep adding more and more desirable constraints

\begin{itemize}
\item ``Not only would I like to stand next to a friend, but Mike makes me look better than Joe.''
\item ``This robot should not do the drilling task as its driller is almost broken.''
\item ``This module should be deployed to server $A$ as its required dependencies are already there.''
\end{itemize}

\vspace*{1ex} \pause 
but at some point, problems become \alert{unsatisfiable}.

\vspace*{2ex} \pause 

\begin{itemize}
\item[-] $\rightarrow$ returning \texttt{UNSAT} is not an option in autonomic computing settings! \pause 
\item[-] We want \emph{graceful} degradation: satisfy most 
constraints (maybe even distinguish them)
\end{itemize}

\end{frame}

\begin{frame}
\frametitle{Over-Constrained Problems}

\vspace*{1ex}

Consider: 

\vspace*{2ex}

$x, y, z \in \{1,2,3\}$ with 
\bgroup\abovedisplayskip4pt\belowdisplayskip4pt
\begin{align*}
  \mathrm{c}_1 &: x + 1 = y
\\[-.4ex]
  \mathrm{c}_2 &: z = y + 2
\\[-.4ex]
  \mathrm{c}_3 &: x + y \leq 3
\end{align*}
\egroup

\begin{itemize}
  \item Not all constraints can be satisfied simultaneously 
\begin{itemize} \pause
  \item e.\,g., $\mathrm{c}_2$ forces $z = 3$ and $y = 1$, making $\mathrm{c}_1$ impossible
\end{itemize}

  \item We \cemph{choose} between solutions that satisfy either $\{ \mathrm{c}_1, \mathrm{c}_3 \}$ or $\{ \mathrm{c}_2, \mathrm{c}_3 \}$.
\end{itemize}

\vspace*{2ex}

Which solutions $v \in [X \to D]$ are \alert{preferred}?

\end{frame}

\begin{frame}
\frametitle{Constraint Relationships}

Ansatz~\cite{Schiendorfer13}
\begin{itemize}
  \item Definiere Relation $R$ über Constraints $C$ um anzugeben, welche Constraints wichtiger sind als andere, e.\,g.
\begin{itemize}
  \item $c_1$ wichtiger als $c_2$

  \item $c_1$ wichtiger als $c_3$
\end{itemize}

\vspace*{2ex}
\item Wie \textbf{viel} wichtiger soll $c_1$ sein?
\begin{itemize}
\item Wichtiger als nur $c_2$ oder $c_3$ allein?
\item Wichtiger als $c_2$ und $c_3$ \emph{zusammen}?
\end{itemize}
\end{itemize}


\begin{textblock*}{2.5cm}[1,1](\textwidth-1.5cm,\textheight-2.03cm)
\begin{tikzpicture}[auto,
                    ->,>=stealth',shorten >=1pt,thick,
                    node distance=.7cm,inner sep=2pt,
                    constraint/.style={circle,fill=black!15,draw,font=\sffamily\small}]
\node[constraint node] (1) at (0, 0)                   {$\mathrm{c}_1$};
\node[constraint node] (2) at ($ (1) + (-0.8, -0.8) $) {$\mathrm{c}_2$};  
\node[constraint node] (3) at ($ (1) + ( 0.8, -0.8) $) {$\mathrm{c}_3$};  
%  
\path[every node/.style={font=\sffamily\tiny}]
  (2) edge (1)
  (3) edge (1)
  ;
\end{tikzpicture}
\end{textblock*}

%\vspace*{2ex}
%\begin{small}
%A.~Schiendorfer, J.-Ph.~Steghöfer, A.~Knapp, F.~Nafz, W.~Reif (2013)
%\end{small}
\end{frame}

\begin{frame}[fragile]{MiniBrass}

\begin{center}

\includegraphics[width=.5\textwidth]{img/minibrass.png}

\vspace*{2ex}

\url{http://isse-augsburg.github.io/constraint-relationships/}

\end{center}

\end{frame}

\begin{frame}
\frametitle{Single-Predecessor-Dominance (SPD)}

\cemph{isWorseThan}-Relation für Mengen verletzter Constraints
%
%\bgroup\abovedisplayskip=4pt\belowdisplayskip=12pt
%\begin{gather*}
%  V \SPDrel{R} V \uplus \{ c \} 
%\\
%  V \uplus \{ c \} \SPDrel{R} V \uplus \{ c' \}
%\quad\text{if $c \rightarrow_R c'$}
%\end{gather*}
%\egroup

\begin{center}
\begin{tikzpicture}[auto,
                    ->,>=stealth',shorten >=1pt,thick,
                    node distance=1cm,inner sep=2pt,
                    constraint/.style={circle,fill=black!15,draw,font=\sffamily\small}]
\begin{scope}
\node[constraint,fill=green] (1) at (0, 0)                   {$\mathrm{c}_7$};
\node[constraint,fill=green] (2) at ($ (1) + (-1.2, -0.8) $) {$\mathrm{c}_4$};  
\node[constraint,fill=red]   (3) at ($ (2) + (0.0, -0.9) $) {$\mathrm{c}_5$};  
\node[constraint,fill=green] (4) at ($ (3) + ( 0.0, -1.0) $) {$\mathrm{c}_6$};
\node[constraint,fill=green] (7) at ($ (1) + ( 1.2, -0.8) $) {$\mathrm{c}_1$};  
\node[constraint,fill=red]   (8) at ($ (7) + (-0.6, -0.9) $) {$\mathrm{c}_2$};  
\node[constraint,fill=red]   (9) at ($ (7) + ( 0.6, -0.9) $) {$\mathrm{c}_3$};  
%  
\path[every node/.style={font=\sffamily\tiny}]
  (2) edge (1)
  (7) edge (1)
  (3) edge (2)
  (4) edge (3)
  (8) edge (7)
  (9) edge (7)
  ;
\end{scope}
%
\begin{scope}[xshift=6cm]
\node[constraint,fill=green] (1) at (0, 0)                   {$\mathrm{c}_7$};
\node[constraint,fill=red]   (2) at ($ (1) + (-1.2, -0.8) $) {$\mathrm{c}_4$};  
\node[constraint,fill=green] (3) at ($ (2) + (0.0, -0.9) $) {$\mathrm{c}_5$};  
\node[constraint,fill=green] (4) at ($ (3) + ( 0.0, -1.0) $) {$\mathrm{c}_6$};  
\node[constraint,fill=green] (7) at ($ (1) + ( 1.2, -0.8) $) {$\mathrm{c}_1$};  
\node[constraint,fill=red]   (8) at ($ (7) + (-0.6, -0.9) $) {$\mathrm{c}_2$};  
\node[constraint,fill=red]   (9) at ($ (7) + ( 0.6, -0.9) $) {$\mathrm{c}_3$};  
%  
\path[every node/.style={font=\sffamily\tiny}]
  (2) edge (1)
  (7) edge (1)
  (3) edge (2)
  (4) edge (3)
  (8) edge (7)
  (9) edge (7)
  ;
\end{scope}
%
\node (R) at (3.05, -1.4) {$\SPDrel{R}$};
\end{tikzpicture}
\end{center}
%
%\cemph{Ordnungs}relation über Zuweisungen 
%\bgroup\abovedisplayskip4pt
%\begin{equation*}
%  w \SPDord{R} v \iff \{ c \in C \mid v \not\models c \} \mathrel{({\SPDrel{R}})^{+}} \{ c \in C \mid w \not\models c \}
%\end{equation*}
%\egroup

\end{frame}


\begin{frame}
\frametitle{Transitive-Predecessors-Dominance (TPD)}

\cemph{isWorseThan}-Relation für Mengen verletzter Constraints
%
%\bgroup\abovedisplayskip=4pt\belowdisplayskip=12pt
%\begin{gather*}
%  V \TPDrel{R} V \uplus \{ c \} 
%\\
%  V \uplus \{ c_1, \ldots, c_k \} \TPDrel{R} V \uplus \{ c' \}
%\quad\text{if $\forall c \in \{ c_1, \ldots, c_k \} \,.\, c \rightarrow_{R}^{+} c'$}
%\end{gather*}
%\egroup

\begin{center}
\begin{tikzpicture}[auto,
                    ->,>=stealth',shorten >=1pt,thick,
                    node distance=1cm,inner sep=2pt,
                    constraint/.style={circle,fill=black!15,draw,font=\sffamily\small}]
\begin{scope}
\node[constraint,fill=green] (1) at (0, 0)                   {$\mathrm{c}_7$};
\node[constraint,fill=green] (2) at ($ (1) + (-1.2, -0.8) $) {$\mathrm{c}_4$};  
\node[constraint,fill=red]   (3) at ($ (2) + (0.0, -0.9) $) {$\mathrm{c}_5$};  
\node[constraint,fill=red] (4) at ($ (3) + ( 0.0, -1.0) $) {$\mathrm{c}_6$};
\node[constraint,fill=green] (7) at ($ (1) + ( 1.2, -0.8) $) {$\mathrm{c}_1$};  
\node[constraint,fill=red]   (8) at ($ (7) + (-0.6, -0.9) $) {$\mathrm{c}_2$};  
\node[constraint,fill=red]   (9) at ($ (7) + ( 0.6, -0.9) $) {$\mathrm{c}_3$};  
%  
\path[every node/.style={font=\sffamily\tiny}]
  (2) edge (1)
  (7) edge (1)
  (3) edge (2)
  (4) edge (3)
  (8) edge (7)
  (9) edge (7)
  ;
\end{scope}
%
\begin{scope}[xshift=6cm]
\node[constraint,fill=green] (1) at (0, 0)                   {$\mathrm{c}_7$};
\node[constraint,fill=red]   (2) at ($ (1) + (-1.2, -0.8) $) {$\mathrm{c}_4$};  
\node[constraint,fill=green] (3) at ($ (2) + (0.0, -0.9) $) {$\mathrm{c}_5$};  
\node[constraint,fill=green] (4) at ($ (3) + ( 0.0, -1.0) $) {$\mathrm{c}_6$};  
\node[constraint,fill=green] (7) at ($ (1) + ( 1.2, -0.8) $) {$\mathrm{c}_1$};  
\node[constraint,fill=red]   (8) at ($ (7) + (-0.6, -0.9) $) {$\mathrm{c}_2$};  
\node[constraint,fill=red]   (9) at ($ (7) + ( 0.6, -0.9) $) {$\mathrm{c}_3$};  
%  
\path[every node/.style={font=\sffamily\tiny}]
  (2) edge (1)
  (7) edge (1)
  (3) edge (2)
  (4) edge (3)
  (8) edge (7)
  (9) edge (7)
  ;
\end{scope}
%
\node (R) at (3.05, -1.4) {$\TPDrel{R}$};
\end{tikzpicture}
\end{center}

%\cemph{Ordnungs}relation über Zuweisungen
%\bgroup\abovedisplayskip4pt
%\begin{equation*}
%  w \TPDord{R} v \iff \{ c \in C \mid v \not\models c \} \mathrel{({\TPDrel{R}})^{+}} \{ c \in C \mid w \not\models c \}
%\end{equation*}
%\egroup

\end{frame}

% block styles
\tikzstyle{sensor}=[draw, fill=blue!20, text width=5em, 
    text centered, minimum height=2.5em,drop shadow]    
    
\tikzstyle{alg} = [sensor, text width=5em, fill=isseorange!20, 
    minimum height=13em, rounded corners, drop shadow]
\tikzstyle{constraint}=[draw, circle, fill=issegrey!20, text width=1.2em, 
    text centered, minimum height=1.5em,drop shadow]
\tikzstyle{domainstore} = [alg, text width=5em, fill=isseorange!40, 
    minimum height=4em, rounded corners]
\tikzstyle{goodc} = [ForestGreen, font=\bfseries]
\tikzstyle{badc} = [Red, font=\bfseries]
\tikzstyle{okayc} = [LimeGreen, font=\bfseries]
        
\tikzset{
vecArrow/.style={
  thick
  }
}

\tikzset{
    mn/.style={rectangle,rounded corners,draw=black, top color=isseorange!5, bottom color=isseorange!30,
                   very thick, inner sep=\myinnersep*1em, minimum size=3em, text centered, outer sep=0, align=center},
    innernode/.style={mn, text width=3cm,  minimum height=1.5cm,
                      top color=issegrey!20, bottom color=issegrey!60},
    emphnode/.style={innernode, top color=isseorange!30, bottom color=isseorange!70}
}

% Define distances for bordering
\def\blockdist{2.3}
\def\edgedist{2.5}

  \tikzset{
    invisible/.style={opacity=0},
    visible on/.style={alt={#1{}{invisible}}},
    alt/.code args={<#1>#2#3}{%
      \alt<#1>{\pgfkeysalso{#2}}{\pgfkeysalso{#3}} % \pgfkeysalso doesn't change the path
    },
  }
  
\begin{frame}{Traditional Constraint Solving}
\begin{center}
\begin{tikzpicture}
% First row:
 \node (search) [alg]  {Search \phantom{$x = 5$} };
 \path (search.east)+(4.6,0) node (propag) [alg,text width =12em]  {};
 \node[below right] at (propag.north west) {Constraint Store $C$};
 
 \path (propag.west)+(0.8,-1.2) node (c1) [constraint] {$c_1$}; 
 \path (propag.west)+(1.1,-0.2) node (c2) [constraint] {$c_2$}; 
 \path (propag.west)+(2.0,0.4) node (c3) [constraint] {$c_3$}; 
 \path (propag.west)+(3.2,0.7) node (c4) [constraint] {$c_4$};
  
 \path (propag.east)+(-1.2,-1.2) node (domainstore) [domainstore] {Domain Store $(D_x)_{x \in X}$}; 
 
 \path [draw,vecArrow, ->] ([yshift=-2em]search.north east) -- node [above,visible on=<2->] {$x\gets5$} ([yshift=-2em]propag.north west);
 \path [draw,vecArrow, <-] ([yshift=2em]search.south east) -- node [above,goodc,visible on=<4->] {$\top$} ([yshift=2em]propag.south west);
 
 \path [draw, vecArrow, <->] (c1.east) -- node [below,visible on=<3->,goodc] {$\top$} (domainstore.west) ;
 \path [draw,vecArrow, <->] (c2.330) -- node [above right,visible on=<3->,goodc] {$\top$} (domainstore.150) ;
 \path [draw,vecArrow, <->] (c3.290) -- node [right,visible on=<3->,goodc] {$\top$} (domainstore.120) ;
 \path [draw,vecArrow, <->] (c4.south) -- node [right,visible on=<3->,goodc] {$\top$} (domainstore.68) ;
\end{tikzpicture}
\end{center}
\onslide<0>{
\begin{columns}[c] % contents are top vertically aligned
     \begin{column}[c]{7cm} % each column can also be its own 
\begin{itemize}
\item Eine Menge von Erfüllungsgraden, $\mathbb{B} = \{ \bot, \top \}$
\item Eine Kombinationsoperation $\wedge$
\item Ein neutrales Element $\top$
\item Eine partielle Ordnung $(\mathbb{B}, \leq_\mathbb{B})$ mit $\top <_\mathbb{B} \bot$ 
\end{itemize}
\end{column}
     \begin{column}[c]{4.5cm} 
     \end{column} 
\end{columns}
}

\end{frame}

\begin{frame}{Klassisches Constraint-Solving}
\begin{center}
\begin{tikzpicture}
% First row:
 \node (search) [alg]  {Search $x = 5$};
 \path (search.east)+(4.6,0) node (propag) [alg,text width =12em]  {};
 \node[below right] at (propag.north west) {Constraint Store $C$};
 
 \path (propag.west)+(0.8,-1.2) node (c1) [constraint] {$c_1$}; 
 \path (propag.west)+(1.1,-0.2) node (c2) [constraint] {$c_2$}; 
 \path (propag.west)+(2.0,0.4) node (c3) [constraint] {$c_3$}; 
 \path (propag.west)+(3.2,0.7) node (c4) [constraint] {$c_4$};
  
 \path (propag.east)+(-1.2,-1.2) node (domainstore) [domainstore] {Domain Store $(D_x)_{x \in X}$}; 
 
 \path [draw,vecArrow, ->] ([yshift=-2em]search.north east) -- node [above,visible on=<2->] {$y \gets 4$} ([yshift=-2em]propag.north west);
 \path [draw,vecArrow, <-] ([yshift=2em]search.south east) -- node [below,badc,visible on=<4->] {$\bot$} ([yshift=2em]propag.south west);
 
 \path [draw, vecArrow, <->] (c1.east) -- node [below,visible on=<3->,badc] {$\bot$} (domainstore.west) ;
 \path [draw,vecArrow, <->] (c2.330) -- node [above right,visible on=<3->,goodc] {$\top$} (domainstore.150) ;
 \path [draw,vecArrow, <->] (c3.290) -- node [right,visible on=<3->,goodc] {$\top$} (domainstore.120) ;
 \path [draw,vecArrow, <->] (c4.south) -- node [right,visible on=<3->,goodc] {$\top$} (domainstore.68) ;
\end{tikzpicture}
\end{center}
\onslide<5->{
\begin{columns}[c] % contents are top vertically aligned
     \begin{column}[c]{8cm} % each column can also be its own 
\begin{itemize}
\item Eine Menge von Erfüllungsgraden, $\mathbb{B} = \{ \bot, \top \}$
\item Eine Kombinationsoperation $\wedge$
\item Ein neutrales Element $\top$
\item Eine partielle Ordnung $(\mathbb{B}, \leq_\mathbb{B})$ mit $\top <_\mathbb{B} \bot$ 
\end{itemize}
   \end{column} 
     \begin{column}[c]{4.5cm} 
     \end{column} 
\end{columns}
}
\end{frame}

\begin{frame}{Soft-Constraint-Solving}
\begin{center}
\begin{tikzpicture}
% First row:
 \node (search) [alg]  {Search $x = 5$};
 \path (search.east)+(4.6,0) node (propag) [alg,text width =12em]  {};
 \node[below right] at (propag.north west) {Constraint Store $C$};
 
 \path (propag.west)+(0.8,-1.2) node (c1) [constraint] {$c_1$}; 
 \path (propag.west)+(1.1,-0.2) node (c2) [constraint] {$c_2$}; 
 \path (propag.west)+(2.0,0.4) node (c3) [constraint] {$c_3$}; 
 \path (propag.west)+(3.2,0.7) node (c4) [constraint] {$c_4$};
  
 \path (propag.east)+(-1.2,-1.2) node (domainstore) [domainstore] {Domain Store $(D_x)_{x \in X}$}; 
 
 \path [draw,vecArrow, ->] ([yshift=-2em]search.north east) -- node [above,visible on=<2->] {$y \gets 4$} ([yshift=-2em]propag.north west);
 \path [draw,vecArrow, <-] ([yshift=2em]search.south east) -- node [below,okayc,visible on=<4->] {$4$} ([yshift=2em]propag.south west);
 
 \path [draw, vecArrow, <->] (c1.east) -- node [below,visible on=<3->,okayc] {$4$} (domainstore.west) ;
 \path [draw,vecArrow, <->] (c2.330) -- node [above right,visible on=<3->,goodc] {$0$} (domainstore.150) ;
 \path [draw,vecArrow, <->] (c3.290) -- node [right,visible on=<3->,goodc] {$0$} (domainstore.120) ;
 \path [draw,vecArrow, <->] (c4.south) -- node [right,visible on=<3->,goodc] {$0$} (domainstore.68) ;
\end{tikzpicture}
\end{center}
\onslide<5->{
\begin{columns}[c] % contents are top vertically aligned
     \begin{column}[c]{8cm} % each column can also be its own environment
    \begin{itemize}
\item Eine Menge von Erfüllungsgraden, z.B., $\{ 0, \ldots, k \}$
\item Eine Kombinationsoperation $+$
\item Ein neutrales Element $0$
\item Eine partielle Ordnung $(\mathbb{N}_0, \geq)$ mit $0$ als Top 
\end{itemize}
   \end{column} \pause
     \begin{column}[c]{4.5cm} 
    %    Eine \cemph{valuation structure}~\cite{Schiex1995valued}, wenn die Ordnung total ist, sonst eine  \cemph{partial valuation structure}~\cite{Gadducci2013} (PVS).
     \end{column}
\end{columns}
    
}
\end{frame}

\begin{frame}{Partial Valuation Structures}
Zugrundeliegende \alert{algebraische Struktur}:
Partial valuation structure \emph{(= partiell geordnetes, kommutatives Monoid)}
\begin{itemize}
\item $(M, \cdot_M, \varepsilon_M, \leq_M)$ 
\item $m \cdot_m \varepsilon_M = m$
\item $m \leq_M \varepsilon_M$
\item $m \leq_M n \rightarrow m \cdot_M o \leq_M n \cdot_M o$
\end{itemize}

\vspace*{2ex}

\begin{columns}[onlytextwidth,T]
    
    \begin{column}{.7\textwidth}
          
    \hSecond{Abstrakt}
    
    \begin{itemize}
    \item $M$ \ldots Elemente
    \item $\cdot_M$ \ldots Kombination von Bewertungen
    \item $\varepsilon_M$ \ldots neutrales, ``bestes'' Element
    \item $\leq_M$ \ldots Ordnung, links ``schlechter''
    \end{itemize}
    \end{column}
    
    \begin{column}{.3\textwidth}
  	\hFirst{Konkret}  
  	 \begin{itemize}
    \item $\{0, \ldots, k \}$ 
    \item $+_k$
    \item $0$ 
    \item $\geq$
    \end{itemize}
    \end{column}
  \end{columns}

  \vspace*{2ex}
  
  \hfill \emph{\cite{Gadducci2013,SchiendorferPvs2015}}
\end{frame}

\begin{frame}[fragile]{SoftConstraints in MiniZinc}
\begin{lstlisting}
% X: {x,y,z} D_i = {1,2,3}, i in X
%    * c1: x + 1 = y   * c2: z = y + 2 * c3: x + y <= 3
% (c) ISSE
% isse.uni-augsburg.de/en/software/constraint-relationships/
include "soft_constraints/minizinc_bundle.mzn";

var 1..3: x; var 1..3: y; var 1..3: z;

% read as "soft constraint c1 is satisfied iff x + 1 = y"
constraint x + 1 = y <-> satisfied[1];
constraint z = y + 2 <-> satisfied[2];
constraint x + y <= 3 <-> satisfied[3];

% soft constraint specific for this model
nScs = 3; nCrEdges = 2;
crEdges = [| 2, 1 | 3, 1 |]; % read c2 is less important than c1

solve minimize penSum; % minimize the sum of penalties
\end{lstlisting}

\end{frame}

\begin{frame}[fragile]{SoftConstraints in MiniBrass}
\begin{lstlisting}
type ConstraintRelationships = PVSType<bool, set of 1..nScs> = 
  params { 
    array[int, 1..2] of 1..nScs: crEdges;
    bool: useSPD;
  } in 
  instantiates with "mbr_types/cr_type.mzn" {
    times -> link_booleans;
    is_worse -> is_worse_cr;
    top -> {};
};  
PVS: cr1 = new ConstraintRelationships("cr1") {
   soft-constraint c1: x < 500;
   soft-constraint c2: x > 500;
   
   crEdges : |- [| c1, c2 |] -|;
   useSPD: false ;
};
solve cr1; 
\end{lstlisting}
\end{frame}