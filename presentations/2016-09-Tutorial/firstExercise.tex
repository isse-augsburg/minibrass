\begin{frame}{MiniZinc: HelloWorld}
\textbf{Exercise 1}

Build a MiniZinc model \texttt{xopt.mzn} with a decision variable $x$ taking values from 0 to 10, with
constraints to ensures that $x$ is divisible by 4, which outputs the value of $x$ that gives the minimum
value of $(x-7)^2$.

Test it using the precompiled IDE-bundle. Suppose you cannot use the \texttt{mod} function, how would you alternatively model that $x$
is divisible by 4?

\end{frame}

\begin{frame}{MiniZinc: Arrays}
\textbf{Exercise 2}
Define a MiniZinc model \texttt{array.mzn} which takes an integer parameter $n$ defining the length of an
array of numbers $x$ taking values from 0 to 9. Constrain the array so the sum of the numbers in
the array is equal to the product of the numbers in the array. Output the resulting array.
Test your model using the ``all solutions'' setting active in the IDE.

\vspace*{2ex}

 Add a constraint to ensure that the numbers in the array are non-decreasing, i.e. $x[1] \leq x[2] \leq
\ldots \leq x[n]$. This should reduce the number of similar solutions. 
How big a number can you solve with your model? Why do you think this happens?


\end{frame}

