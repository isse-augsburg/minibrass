% a sequence (list), e.g., values [1, 2, 3]
\providecommand{\seq}[1]{\langle #1 \rangle}

% a concrete variable (in an example)
\providecommand{\concVar}[1]{\mathtt{#1}}

% just shorthand for typing set braces
\providecommand{\set}[1]{\{ #1 \}}

% --------------------------------------------------
% Elementary definitions
\providecommand{\Vars}{X}
% a generic variable for iterating etc
\providecommand{\GenVar}{x}
\providecommand{\Dom}{D}
\providecommand{\Cons}{C}

% problem identifier
\providecommand{\Prob}{\mathit{CP}}

% a generic constraint
\providecommand{\GenCons}{c}

% macro for the search space
\providecommand{\SSpace}{[\Vars \to \Dom]}

% macro for the set of boolean values
\providecommand{\Bool}{\mathbb{B}}

% a generic objective function
\providecommand{\GenObj}{f}

% --------------------------------------------------

% a generic element of a meet monoid / PVS / c-semiring describing satisfaction, i.e., the "solution degree"
\providecommand{\GenSolDegree}{m}

% a generic variable assignment, i.e., an element taken from [X \to D]
\providecommand{\GenAssignment}{\theta}

% the solution operator giving the set of assignments that satisfy all hard constraints of a problem P
\providecommand{\solns}{\mathrm{sol}}

% the operator mapping assignments to their solution degrees
\providecommand{\val}{\mathrm{val}}

% --------------------------------------------------- 
% CATEGORICAL notation

% categories
\providecommand{\category}[1]{\mathrm{#1}}
\providecommand{\setcat}{\category{Set}}
\providecommand{\POcat}{\category{PO}}
\providecommand{\uSLcat}{\category{uSL}}
\providecommand{\pocMoncat}{\category{pocMon}}
\providecommand{\jMoncat}{\category{jMon}}
\providecommand{\mMoncat}{\category{mMon}}
\providecommand{\xMoncat}{{x}\category{Mon}}
\providecommand{\cpocMoncat}{\category{cpocMon}}
\providecommand{\imMoncat}{\category{imMon}}
\providecommand{\cmMoncat}{\category{cmMon}}
\providecommand{\cSRngcat}{\category{cSRng}}
\providecommand{\DAGcat}{\category{DAG}}

% functors
\providecommand{\functor}[1]{\mathit{#1}}
\providecommand{\idfun}[1]{\mathrm{id}_A}